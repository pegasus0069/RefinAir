\chapter*{\centering Abstract}

The main objective of this project is to eliminate the climate change crisis around the world, by fulfilling the Sustainable Development Goal 13 which is Climate Action. The data we are extracting from this project about polluting compounds will be an extended research, which we believe will help us to find a solution to this worldwide crisis and spread awareness globally. 

For the time being, we are proposing two solutions to this global problem. We propose to extract data from a primary source which we aspire to construct using IoT and a secondary source which is AQUA Satellite. For the primary source, we will demonstrate an IoT based device and compare it with the expensive industrial graded air monitoring device, AirVisual Pro by IQAir, to determine device compatibility and try to find an inexpensive solution to the problem. Since we have to make sure to measure the air quality of all the regions of Bangladesh, the inexpensive device that we will be making will no longer be cost efficient since we need to fill the infrastructure gap. To solve this problem, a successful approach has been noted from multiple research where the AOD PM2.5 product is predicted using Machine Learning Algorithms (MLAs). In the process of making an atmospheric forecast reporting system, we intend to incorporate station wise graphs and using MLAs our aim is to generate an ‘AQI alert atmospheric map’,with the help of the satellite data. Additionally, our project will extensively monitor the most polluting areas/routes - to give an alert when the pollution exceeds the favourable amount. 

For the primary data source, our project aims to distinguish the particle concentration of PM1.0, PM2.5, PM10, CO, NO2, CO2 and VOCs like (CH4,C3H8,C4H10,C2H5OH). For detecting PM1.0, PM2.5 and PM10, PMS5003 PM2.5 Air Particle Dust Sensor is used and the NO2and VOCs like (CH4,C3H8,C4H10,C2H5OH) compounds are measured by Grove Multichannel Gas Sensor. Since increment of concentration of CO is becoming a prime concern in Bangladesh, we have used a sensor called MQ 7 for the detection of CO -  for more accuracy of the data. Similarly, for CO2concentration observation, MH-Z19B Co2 Sensor Module is chosen to work with. Not to keep our system limited to only to the extraction of polluting compound data, we used the BME280 module to keep track of the temperature, humidity and barometric pressure. Since this is an IoT based air monitoring system, the user will be able to get all the information regarding the air quality.

In order to make the IoT based air monitoring system, Arduino Mega 2560 Wifi - which is a development board- is used as the microcontroller of the system and incorporated with the set of sensors to detect the concentration of pollutants and monitor the data patterns. It is turned to an IoT based device after the set of sensors are connected to the development board. Arduino Mega 2560 WiFi takes readings from the sensors and sends all the sensor data to the PC using Serial Communication. This data is then migrated to the cloud. All the concentration of polluting compound charts will be shown in a Google Sheet visualization.

The secondary AOD(Aerosol Optical Depth)  550nm data is aimed to be extracted from AQUA Satellite that will be later collected using NASA Giovanni data visualization platform. The available AOD 550 nm product needs to be preprocessed from MODIS (Moderate Resolution Imaging Spectroradiometer) instrument aboard the Aqua Satellite, which passes from South to North poles of the earth. The selected shape of the data is Bangladesh, which means the obtained AOD 550 nm product will be collected and preprocessed by the AQUA satellite using the MODIS instrument while orbiting above Bangladesh. The obtained AOD 550 nm is the mean AOD 550 nm for each day processed by the AQUA MODIS.



