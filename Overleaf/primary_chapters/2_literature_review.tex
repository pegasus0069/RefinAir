\chapter{Literature Review}
This section of the report will consist of literature reviews on the two propositions we aim to execute to diminish and monitor the degradation of air quality. In order to fulfil the SDG, we proposed an IoT based inexpensive solution to solve the problem. It has been noticed that we have to create IoT based devices in industry in order to be able to monitor the air quality of all the regions of Bangladesh. Instead of adding up another cause of pollution, we came up with another promising proposition where we will extract data from AQUA satellite and use MLAs to estimate the AOD PM2.5 product. Our aim is to make an atmospheric forecast reporting system with station wise graphs. 

The primary data here comes from the primary source,IoT based device, and the secondary data comes from the secondary source, AQUA Satellite. However, the literature review below is divided into two sections for the two different sources of data.

\section{Literature Review on IoT Based Device}

The following sensor system assures air quality monitoring via an array of sensors which transmit their reading through Bluetooth to the nearest smartphone. The readings get updated every time the application- installed in the smartphone- is clicked. This system also has the facility to track the location of the set of sensors by GPS in google maps. This function of the embedded system is controlled by NXP ARM mbed LPC1768 which is a microcontroller and RN-42 bluetooth module is used for communication. The advantages of this system are that this system has a wide range of toxic gases that can be detected within just a click and the GPS tracker helps more in mobile monitoring. The drawbacks of this system are that this system allows mobile monitoring but it does not have a database of data which can be later used for taking necessary measures and the bluetooth, RN-42 module, might not work efficiently if the user is out of its range and still wants to get the readings. (Yang & Li, 2015, #)

This smart vehicle monitoring system for air pollution detection using Wsn is a system for the moving vehicles to monitor the NO2, Humidity, Temperature, CO levels of air contamination by using NO2 sensor, Humidity sensor, Temperature sensor, CO sensor. MANET(Mobile Ad Hoc Network) routing algorithm is used, which has nearly 28 mobile nodes(Vehicles) that provide a coverage area of 300meters around the city. The sensor data of the vehicles will be sent to the smartphones of the appropriate drivers to monitor. The microcontroller of this system is PIC Microcontroller 16f877a and for communication Zigbee is used. This system has a high Coverage of areas with the help of 28 moving vehicles with a range of 300meters per vehicle, thus collecting a large range of data. By using a cloud network, large amounts of data of different vehicle records can be stored and retrieved for future purposes. Mobile monitoring is another plus point of this system. But this device can only detect some toxic gases and does not have a GPS facility, which can be counted as pitfalls of this system. (Suganya & Vijayashaarathi, 2016, #)

In this existing system, which is known to develop Arduino based embedded systems to detect toxic gases has an array of sensors are set within some parameters to detect the concentration of toxic gases and vapors in air. The parameters for the sensors will be set inside the code.If the parameters are breached, the system switches on the buzzer and red LED and outputs an alarm message to the LCD module and PC via the serial interface. Also sends SMS to the mobile device via a GSM module. Only few MQ sensors are used here and are incorporated with ATmega2560 which is the  microcontroller of this system and for communication GSM module is used. But this system gives only real-time data, thus does not have a database which can be used for future purpose and the data can not be shown anytime on smartphones unless there’s a breach, thus partial mobile monitoring. (Holovaty et al., 2018, #)

Another project about low cost IoT based air monitoring system using Raspberry Pi with MySQL Database is reviewed below. Here, a set of data is extracted from some sensors which had some threshold values set by the system. Raspberry-Pi is interfaced with various sensors (temperature, Humidity, MQ 5 Gas Sensor) and real-time data will be obtained and stored in MySQL Database. If the sensor value from the environment exceeds the threshold value, the communication module sends the message alert to the client by using ThinkSpeak open data IoT platform. In this system, NodeMCU and Raspberry Pi are used as microcontrollers and ESP8266 is used for data transmission. All the data is stored for future research purposes. However, this device has less variety of sensors and does not have any facility for sending SMS to the user in case the user does not have internet and the thresholds are breached. No location trackers are incorporated in this system which can be known as another drawback of this system. (Kiruthika & Umamakeswari, 2017, #)

The following literature review is about a project where a mobile GPRS-Sensors array is used for air pollution monitoring. They have developed an air pollution geo-sensor network Consisting of 24 sensors taking 24/7 readings of CO, NO2 and SO2 to obtain the AQI (Air Quality Index). They've used GPRS-Modem to transmit the data to a php and mysql based database server and GPS-Module to locate the location of the sensors and using google maps APIs they have obtained the real time readings of the sensors and the current location of sensors. They've developed a system which can be portable and can be attached to any public transport to obtain sensor readings remotely with live GPS location. The sensors are incorporated with HCS12 which is a microcontroller in this system. The collected data is stored for future research purposes and helps achieve statistical graphs to track the sensor readings. This method provides real-time location and sensor readings of the sensor array. Additionally, this device is portable and can be attached to any public transport to obtain sensor readings and GPS location remotely. However, they could’ve used a single microcontroller board with embedded GPS and GSM to make the system more compact and no sensor to detect PM2.5 concentration was used. (Al-Ali et al., 2010, #)

In this research, IoT is used for Mobile–Air pollution monitoring system where an IoT solution is developed to detect the AQI (Air Quality Index) and the concentration of Carbon Monoxide (CO), Methane (CH4) and Carbon dioxide (CO2) gases. They also used GPS to get the location and timestamp of the devices to get the sensor readings of the specific locations. They used an ESP8266 to send the obtained sensor readings and the GPS location to a cloud server. They have used Ubidots as a cloud server to receive the data from the devices remotely and then send the data to the android app they’ve built. In the android app the user can see the pollution level of any specific route the user is using. ATmega328P and ESP8266 are incorporated in this system as a microcontroller and communication purpose simultaneously. The collection of CO, CH4, CO2 concentration and AQI data is stored for future research purposes, but no sensor for the detection of PM2.5 compounds is used here. This device is portable and can be attached to any public transport to obtain sensor readings and GPS location remotely. (Dhingra et al., 2019, #)

\section{Literature Review on AOD PM2.5 Product Prediction Analysis using Machine Learning Algorithms}

A hybrid remote sensing and machine learning approach, named RSRF model is proposed to estimate daily ground-level PM2.5 concentrations, which integrates Random Forest (RF), one of machine learning (ML) models, and aerosol optical depth (AOD), one of remote sensing (RS) products. The model inputs consist of data sets for AOD, SO2, NO2, CO, O3, AT, RH,WS, P and WD. The RSRF model was compared with Multiple Linear Regression (MLR), Motivation, Abilities, Role Perception and Situational Factors (MARS) and Support Vector Machine (SVR) models. The RSRF, MARS and SVR models have higher prediction accuracies than the MLR model. The results indicated that the RSRF model had a relatively high prediction accuracy, outperforming other three models. The main conclusions derived for this study is that the prediction abilities of different predictors on PM2.5 concentrations vary seasonally. AOD is not a unique indicator for fine particle pollution. As a result, more directly relevant predictors such as important precursor pollutants should be considered. Although weather variables have less direct effects on PM2.5 pollution, they are essential in the RSRF model. (Li & Zhang, 2019, #)

In China, a geo-intelligent deep learning model, Geoi-DBN, was developed for better representation of  the AOD-PM2.5 relationship using geographical distance and spatiotemporally correlating to PM2.5 in a deep belief network.. The geographical correlation was adopted to significantly improve the estimation accuracy. Geoi-DBN can capture the essential features associated with PM2.5 from latent factors. The results show that Geoi-DBN performs significantly better than the traditional neural network. However, in this research  the Geoi-DBN achieved satisfactory performance and in the Geoi-DBN cross-validation slope of observed PM2.5 versus estimation indicated some evidence of bias. The reason behind this was groundlevel PM2.5 is greater than 60 μg/m3 in China. Additionally, this underestimation may be down to several reasons, including the possibility of mixed types and layers of aerosols in the atmosphere and the hygroscopicity of urban aerosols. While previous studies have used machine learning to simulate the AOD-PM2.5 relationship, this study further considers the geographical correlation to greatly improve model performance. (Li et al., 2017, #)

For mountainous regions, a recently developed algorithm, Multiangle Implementation of Atmospheric Correction (MAIAC) is used for the Moderate Resolution Imaging Spectroradiometer (MODIS), which provides Aerosol Optical Depth (AOD) at a high resolution of 1 km. A filtering scheme has been developed to reduce the two main sources of artifacts in MAIAC high resolution AOD from clouds and snow. MAIAC AOD has similar accuracy as MODIS Collection 5.1 AOD product but provides information at 2–3 km spatial scale and with better data coverage due to the higher resolution and less restrictive statistical filtering. The MAIAC AOD product can be used for climate related studies, i.e. the assessment of seasonal or annual averages. (Emili et al., 2011, #)

In a statistical model, that is trained to predict hourly concentrations of particles smaller than 10 m (PM10) by combining satellite-borne Aerosol Optical Depth (AOD) with meteorological and land-use parameters, is shown that besides human emissions, concentrations of particles in the air are to a large extent driven by meteorological factors such as wind direction.With increasing data availability and computational power, machine learning methods, e.g. Artificial Neural Networks  and Random Forests (RF) have been applied frequently in recent years. These machine learning models are beneficial as they efficiently reproduce non-linear relationships and interactions of input features.To this end, Gradient Boosted Regression Trees (GBRT) are used to understand and quantify the conditions driving air quality, as well as determinants of the relationship between AOD and PM10. GBRT has been successfully applied to study sensitivities of aerosol processes before.GBRT as implemented in python’s scikit-learn module are used and merges several statistical approaches found in machine learning applications like decision trees, boosting and with gradient descent. The use of GBRT proved fruitful to understand interconnected processes and the approach presented here can be potentially expanded to other research questions focusing on the understanding of multivariate processes. Future e reports will further address the determination of mechanisms leading to high pollution events using machine learning not only for total PM10 concentrations, but for individual aerosol species. (Stirnberg et al., 2020, #)

In this paper, an improved high-spatial-resolution aerosol retrieval algorithm with land surface parameter support (I-HARLS) at 1-km resolution for MODIS images is developed. A precalculated global land surface reflectance (LSR) database is constructed using the MODIS 8-day synthetic surface reflectance (MOD09A1) products, and a prior seasonal global Land Aerosol Type (LAT) database is created using the MOD04 daily aerosol products. The main aerosol optical properties and types are determined based on the monthly average historical aerosol optical properties from local AERosol RObotic NETwork (AERONET) sites. For cloud screening, the Universal Dynamic Cloud Detection Algorithm (UDTCDA) is selected to mask cloud pixels in remote sensing images. Then, a 1-km-resolution AOD dataset is generated based on the I-HARLS algorithm. Successful AOD retrievals are available over dark and bright surfaces. To test and validate the performance of the I-HARLS algorithm, four typical regions (including Europe, North America, Beijing-Tianjin-Hebei and the Sahara) with different underlying surface and aerosol types are selected for aerosol retrieval experiments. Moreover, AERONET Version 2 Level 2.0 AOD measurements and MODIS daily AOD products at 3-km resolution (MOD04 3K) are selected for validations and comparisons.The results show that the I-HARLS algorithm performs well overall at both the site and regional scales, and AOD retrievals are highly correlated with AERONET AOD measurements.This study shows that although the new AOD retrieval algorithm performs well overall over land, certain problems remain. However, due to the long time series of MODIS data records, longer and wider-scale experiments and validations need to be undertaken. In addition, this paper only performs comparisons with current operational and free-open high-resolution MOD04 aerosol products; therefore, more comprehensive and effective comparison efforts with other high-resolution products (such as MAIAC products) need to be performed in future studies. (Wei et al., 2018, #)

This study aimed to develop machine learning-based models for predicting hourly street-level PM2.5 and NOx concentrations at three roadside stations in Hong Kong. This study highlights the capability of MLAs to produce high temporal resolution air pollution predictions, which can supplement traditional methods (e.g., land use regression) in generating accurate and high-temporal-resolution estimations of air pollution concentration.The researchers comprehensively evaluated and compared the performance of six common machine learning algorithms (MLAs) including Random Forest (RF), Boosted Regression Trees (BRT), Support Vector Machine (SVM), Extreme Gradient Boosting (XGBoost), Generalized Additive Model (GAM), and Cubist and hence applied the most suitable MLAs to apportion the contributions from emission and non-emission factors to hourly street-level PM2.5 and NOx concentrations. The results show that RF outperformed other MLAs and BRT, XGBoost and Cubist presented comparable predictive performances. SVM and GAM have worse predictions than other MLAs. (Li et al., 2020, #)

 
